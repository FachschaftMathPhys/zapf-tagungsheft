% !TEX ROOT = main.tex

\usepackage[ngerman]{babel}
\usepackage[utf8]{inputenc}
\usepackage{libertine}
\usepackage{libertinust1math}
\usepackage[T1]{fontenc}
\usepackage{microtype}
\usepackage{amsmath}
\usepackage{eurosym}
\usepackage[pdftex]{graphicx}
\usepackage{fancyhdr}
\usepackage{blindtext}
\usepackage{todonotes}
\usepackage{geometry}
\usepackage{url}
\usepackage{fontawesome}
\usepackage{hyperref}
\usepackage{wrapfig}
\usepackage{pdfpages}
\usepackage{tocloft}
\usepackage{subcaption}
\usepackage{paralist}
\usepackage{float}
\usepackage{afterpage}

% Suche nach Grafiken in ./media und .:
\graphicspath{{./media/}{./}}

% Satzspiegel
\geometry{papersize={154mm,216mm}, layout=a5paper, layouthoffset=3mm,     layoutvoffset=3mm, inner=20mm, outer=15mm, top=15mm, bottom=25mm, heightrounded, marginparwidth=37mm, marginparsep=5mm}
%\setlength{\parindent}{0pt}
    
%Headlines
\pagestyle{fancy}
\fancyhf{}
\renewcommand{\headrulewidth}{0pt}
%\fancyhead[LE]{\leftmark}
%\fancyhead[RO]{\rightmark}
\fancyfoot[RO,LE]{\thepage}

%Lied-Umgebung einspaltig
\newenvironment{lied}[2]%
  {\section*{#1}\textit{#2}\\\\\bgroup\footnotesize}
  {\egroup\vspace{1cm}}

%Lied-Umgebung zweispaltig mit einspaltigem Titel 
\newenvironment{lied*}[2]%
  {\twocolumn[\section*{#1}\textit{#2}\\]\bgroup\footnotesize}
  {\egroup\vspace{1cm}}  

%Lied-Umgebung komplett zweispaltig oder einspaltig jenachdem wie es davor definiert war
\newenvironment{lied**}[2]%
  {\section*{#1}\textit{#2}\\\\\bgroup\footnotesize}
  {\egroup\vspace{1cm}}    
  
%Lied-Umgebung zweispaltig ohne Untertitel
\newenvironment{lied***}[1]%
  {\twocolumn\section*{#1}\bgroup\footnotesize}
  {\egroup\vspace{1cm}}  
  
%Unterdrückt Section und Subsection Nummern, also werden nur Chapter Nummer angezeigt
  \setcounter{secnumdepth}{0}