% !TEX TS-program = pdflatex
% !TEX encoding = UTF-8 Unicode
% !TEX ROOT = main.tex

%Nutze nicht die automatische Nummerierung
%\renewcommand*\thesection{}
%\renewcommand*\thesubsection{}
%\renewcommand*\thesubsubsection{}

%from https://github.com/ZaPF/Geschaeftsordnung_ZaPF
\section*{Geschäftsordnung für Plenen der ZaPF}

Begriffe und Regelungen, die im Anhang kommentiert oder erklärt werden, sind
kursiv gedruckt.


\subsection*{1 Geltungsbereich%
  \label{geltungsbereich}%
}

Diese Geschäftsordnung gilt für die Plenen (Vollversammlungen aller Teilnehmer)
der Zusammenkunft aller Physikfachschaften (ZaPF).
Sie ist von allen Teilnehmerinnen und Teilnehmern einzuhalten und regelt unter
anderem den Ablauf des Plenums, die Wahl der Organe der ZaPF entsprechend der
Satzung der ZaPF und die Antragsfristen und Abstimmung von Anträgen.

Als teilnehmende Personen der ZaPF gelten alle angemeldeten Teilnehmer und
Teilnehmerinnen der ZaPF, die ihren Tagungsbeitrag entrichtet haben, sowie alle
Mitglieder und Helferinnen und Helfer der ausführenden Fachschaft.


\subsection*{2 Ablauf eines Plenums%
  \label{ablauf-eines-plenums}%
}

\begin{enumerate}
\item Sitzungen der ZaPF sind öffentlich.

\item Die Sitzungsleitung wird von der die ZaPF organisierenden Fachschaft
vorgeschlagen und im Plenum abgestimmt.
Bis zur Wahl der Sitzungsleitung fungiert die ausrichtende Fachschaft als
Sitzungsleitung.

\item Zu Beginn der Sitzung werden ein oder mehrere Protokollführer bzw.
Protokollführerinnen gewählt, das Protokoll der Sitzung wird im
ZaPF-Reader für die folgende ZaPF abgedruckt.

\item Nach der Wahl der Sitzungsleitung und der Protokollführung ist die
Beschlussfähigkeit festzustellen.

\item Anschließend wird die Tagesordnung bekanntgegeben und abgestimmt.
Diese Tagesordnung ist bindend.

\item Im Anfangsplenum werden nach Abstimmung der Tagesordnung die
Vertrauenspersonen gewählt.

\item Im Abschlussplenum sollte es immer einen Tagesordnungspunkt \textquotedbl{}Berichte
der Arbeitskreise\textquotedbl{} geben.
Möchte ein Arbeitskreis (AK) einen Antrag abstimmen bzw. ein Meinungsbild
einholen wollen, so ist diese entsprechend des Abschnittes \textquotedbl{}Anträge\textquotedbl{}
einzureichen.
Auf einer vorherigen ZaPF durch einen GO-Antrag auf \textquotedbl{}Schließung der Redeliste
und Verweisung in eine Arbeitsgruppe mit Recht auf ein Meinungsbild im
Plenum\textquotedbl{} vertagte Anträge sowie solche, die wegen mangelnder
Beschlussfähigkeit, nicht mehr behandelt werden konnten, sollen priorisiert
behandelt werden.

\item Ist in einer Sitzung strittig, wie eine Bestimmung dieser Geschäftsordnung
auszulegen oder wie eine Lücke zu schließen ist, so kann die Auslegungsfrage
mit Wirkung für die gesamte Sitzung durch die Sitzungsleitung entschieden
werden.

\item Die Sitzungsleitung kann die Sitzung unterbrechen, dies sollte in der
Regel jedoch zehn Minuten nicht überschreiten.
\end{enumerate}


\subsection*{3 Anträge%
  \label{antrage}%
}


\subsubsection*{3.1 Antragsfristen und Antragsdurchführung%
  \label{antragsfristen-und-antragsdurchfuhrung}%
}

\begin{enumerate}
\item Antragsberechtigt sind alle teilnehmende Personen.

\item Anträge (z.B. für Tagesordnungspunkte oder Abstimmungen) sind mindestens
eine Stunde vor Beginn des Plenums schriftlich bei der die ZaPF
ausrichtenden Fachschaft einzureichen.
Dies gilt insbesondere für Texte, über die abgestimmt werden soll.
Die Arbeitskreise haben dafür zu sorgen, dass dies rechtzeitig geschieht.
Die Fristen für Anträge zur Änderung der Geschäftsordnung werden in einem
eigenen Absatz geregelt.

\item Anträge, die nach dieser Frist eingereicht werden, sind Initiativanträge
und müssen von mindestens zwei Personen aus verschiedenen Fachschaften
getragen werden. Auch diese Anträge müssen dem Plenum in geeigneter Form
vorgelegt werden.

\item Anträge zur Änderung der Geschäftsordnung zur Abstimmung im Anfangsplenum
müssen mindestens 7 Tage vor dem Anfangsplenum der ZaPF geeignet
bekanntgemacht werden, z.B. über die Mailingliste.
Zur Abstimmung im Zwischen- oder Abschlussplenum müssen Anträge zur Änderung
der Geschäftsordnung spätestens um 15:00 Uhr am Tag vor dem Zwischen- oder
Abschlussplenum bekanntgegeben werden.
Änderungen dieser Geschäftsordnung sind nicht durch Initiativanträge möglich.
Die Änderung der Geschäftsordnung tritt automatisch zum nächsten Plenum in Kraft.

\item Die antragsstellende Person muss im Plenum anwesend sein
oder kann einen Vertreter oder eine Vertreterin benennen und muss dies
der Sitzungsleitung mitteilen.
Die Vertreterin oder der Vertreter ist dann die neue antragstellende Person.

\item Anträge, die bestehende Aussagen der ZaPF, insbesondere die Geschäftsordnung
und die Satzung, ändern wollen, sollen ihre Änderung des bestehenden Textes
\emph{geeignet nachvollziehbar} machen.
Diese Pflicht entfällt für Initiativanträge.
\end{enumerate}


\subsubsection*{3.2 Geschäftsordnungsanträge%
  \label{geschaftsordnungsantrage}%
}

\begin{enumerate}
\item \emph{Geschäftsordnungsanträge} (GO-Anträge) werden durch das Heben
beider Arme signalisiert und sind spätestens vor der nächsten Wortmeldung
bzw. Abstimmung zu behandeln und abzustimmen.

\item Es ist nur eine Für-Rede durch die antragstellende Person und eine Gegenrede
erlaubt, dabei ist eine inhaltliche einer formellen Gegenrede vorzuziehen.
Eine Diskussion von GO-Anträgen findet nicht statt.

\item In der Abstimmung ist (bis auf unten angegebene Ausnahmen) eine einfache
Mehrheit erforderlich.
Gibt es keine Gegenrede gilt der Antrag als angenommen.

\item Geschäftsordnungsanträge sind folgende Anträge:

\begin{itemize}
\item zur Änderung der Tagesordnung,

\item zur erneuten Feststellung der Beschlussfähigkeit
(ohne Abstimmung, ohne Gegenrede),

\item zur Unterbrechung der Sitzung (auch bekannt als \textquotedbl{}Pause\textquotedbl{}),

\item zur Vertagung eines Verhandlungsgegenstandes in einen anderen
Tagesordnungspunkt,

\item zur Begrenzung der Redezeit,

\item zum Schluss der Redeliste (nach Annahme des Antrages können sich
noch Redner auf die Liste setzen lassen, anschließend wird die Liste
geschlossen, weitere Wortmeldungen sind dann nicht mehr möglich)

\item Wiedereröffnung der Redeliste *

\item geschlossene Sitzung (jeweils nur für einen Tagesordnungspunkt)

\item Zulassung Einzelner zur geschlossenen Sitzung

\item zum Schluss der Debatte (die Diskussion wird nach Annahme des
Antrages sofort abgebrochen, eine Abstimmung zum Thema wird ggf.
sofort durchgeführt, auch bekannt als \textquotedbl{}Antrag auf sofortige Abstimmung\textquotedbl{}) *

\item zur Anzweiflung einer Abstimmung (ohne Gegenrede, ohne Abstimmung)

\item zur Schließung der Redeliste und Verweisung in eine Arbeitsgruppe mit
Recht auf ein Meinungsbild im Plenum (auch bekannt als \textquotedbl{}Vertagung auf das
nächste Plenum bzw. die nächste ZaPF\textquotedbl{}) *

\item Nichtbefassung auf dieser ZaPF *

\item geheime Abstimmung (ohne Gegenrede, ohne Abstimmung, setzt namentliche
Abstimmung und Abstimmung per Handzeichen außer Kraft)

\item Neuwahl der Sitzungsleitung unter Benennung eines oder mehrerer Gegenkandidaten

\item Neuwahl des Protokollanten unter Benennung eines oder mehrerer Gegenkandidaten

\item Einholung eines Meinungsbildes im Plenum

\item Verfahrensvorschlag

\item namentliche Abstimmung (ohne Gegenrede, ohne Abstimmung, setzt Abstimmung
per Handzeichen außer Kraft)

\item Abstimmung per Handzeichen (ohne Gegenrede, ohne Abstimmung, nur bei
Abstimmungen und Meinungsbildern)
\end{itemize}

Mit einem * gekennzeichnete Anträge erfordern eine Zweidrittelmehrheit.
\end{enumerate}


\subsection*{4 Abstimmungen und Wahlen%
  \label{abstimmungen-und-wahlen}%
}

Dieser Abschnitt regelt die Abstimmungen und Meinungsbilder des ZaPF-Plenums
sowie die Wahlmodi für Personenwahlen. Die Beschlussfähigkeit für Abstimmungen
und Personenwahlen ist gegeben, wenn \emph{zwanzig Physikfachschaften}
im Plenum anwesend sind.

Die Beschlussfähigkeit ist ausschließlich für Abstimmungen und Personenwahlen
entsprechend dieser Geschäftsordnung notwendig.
Nur das Plenum betreffende Abstimmungen können ohne Beschlussfähigkeit
durchgeführt werden, dies betrifft insbesondere die Wahl der Sitzungsleitung und der
Protokollanten, sowie das Sitzungsende.

Die Sitzungsleitung übt die Funktion des Wahlausschusses für offene Abstimmungen und
Wahlen aus. Für geheime Abstimmungen und Wahlen wird ein Wahlausschuss von der
Sitzungsleitung bestimmt. Hierbei darf kein Mitglied des Wahlausschusses selbst zur
Wahl stehen.


\subsubsection*{4.1 Abstimmungen und Meinungsbilder%
  \label{abstimmungen-und-meinungsbilder}%
}

\begin{enumerate}
\item Es werden Abstimmungen und Meinungsbilder unterschieden. Meinungsbilder
sind informelle Abstimmungen um die Meinung der im Plenum anwesenden
einzuholen, während Abstimmungen über die Annahme oder Ablehnung von
Beschlüssen entscheiden.

\item Beschlüsse sind nach außen zu tragende \emph{Resolutionen}, die zwingend einen
Adressaten haben müssen, \emph{Positionspapiere}, die keinen Adressaten haben,
sowie ZaPF-interne \emph{Selbstverpflichtungen} und Aufträge an den StAPF.

\item Stimmberechtigt für Meinungsbilder ist jede teilnehmende Person der ZaPF.

\item Stimmberechtigt für Abstimmungen ist jede im Plenum anwesende Fachschaft
die mindestens eine teilnehmende Person hat.
Jede Fachschaft hat eine Stimme; wie sie abstimmt, ist innerhalb der
jeweiligen Fachschaft zu regeln.
Den Fachschaften ist Zeit zur Beratung zu gewähren.
Eine geheime Abstimmung ist möglich.

\item Ein Beschluss gilt als angenommen, wenn die Anzahl der Ja-Stimmen größer
ist als die Summe aus Enthaltungen und Nein-Stimmen.
Sollte die Zahl der Enthaltungen die Summe der Ja- und Nein-Stimmen
überwiegen, wird die Abstimmung einmalig wiederholt.
Falls in der erneuten Abstimmung wiederum die Zahl der Enthaltungen
überwiegt, gilt der Antrag als abgelehnt.
Die Abstimmung ist geeignet, z.B. durch deutliches Handheben, kenntlich zu
machen, eine geheime Abstimmung in Papierform kann beantragt werden.
Eine schriftliche Stimmabgabe ist bei vorzeitiger Abreise möglich, es ist
jedoch bei geheimer Abstimmung auf Wahrung des Wahlgeheimnisses zu achten.
Die schriftliche Stimmabgabe gilt nur für inhaltlich unveränderte Anträge
und verfällt sonst.
Stimmrechtsübertragung ist nicht möglich.
Anträge zur Abstimmung sind positiv zu formulieren.

\item Änderungsanträge ändern den Wortlaut eines Antrages, aber nicht das Wesen.
Sie können von jeder teilnehmenden Person gestellt werden.
Änderungsanträge sind vor dem eigentlichen Antrag zu beschließen.
Soweit das Plenum den Änderungsanträgen zustimmt oder sie vom
Hauptantragsteller oder von der Hauptantragstellerin übernommen werden,
wird der Hauptantrag in der geänderten Fassung zur Beschlussfassung gestellt.
Die antragstellende Person hat bis zur endgültigen Beschlussfassung das Recht,
auch eine geänderte Fassung ihres Antrages zurückzuziehen.

\item \emph{Konkurriende Anträge} sind einander widersprechende Anträge zur selben Sache.

\item Bei konkurrierenden Anträgen ist die Beschlussfassung wie folgt durchzuführen:
Geht ein Antrag weiter als ein anderer, so ist über den weitergehenden
zuerst abzustimmen.
Wird dieser angenommen, so sind weniger weit gehende Anträge erledigt.
Lässt sich ein Weitergehen nicht feststellen, so bestimmt sich die
Reihenfolge, in der die konkurrierenden Anträge zur Beschlussfassung
gestellt werden, aus der Reihenfolge der Antragsstellung.
Lässt sich diese nicht mehr feststellen, entscheidet die Sitzungsleitung.

\item Beschlüsse zur Änderung dieser Geschäftsordnung bedürfen einer absoluten
Mehrheit.
Die Geschäftsordnungsanträge, die einer Zweidrittelmehrheit bedürfen, können nur
explizit und mit einer Zweidrittelmehrheit geändert werden.
\end{enumerate}


\subsubsection*{4.2 Personenwahlen%
  \label{personenwahlen}%
}

\begin{enumerate}
\item Das passive Wahlrecht für Personenwahlen haben alle teilnehmenden Personen
der ZaPF. Von dieser Regel wird abgesehen, falls die Personenwahl eine
Wiederwahl oder Bestätigung im Amt ist, so dass in diesem Fall auch nicht
anwesende Teilnehmerinnen und Teilnehmer gewählt werden können.

\item Personenwahlen sind grundsätzlich geheim durchzuführen.
In Abweichung davon dürfen Sitzungsleitung und Protokollführung per
Akklamation gewählt werden.

\item Es werden die Wahlmodi für normale Personenwahlen und die Wahl der
Vertrauenspersonen im Anfangsplenum unterschieden.

\item Stimmberechtigt für normale Personenwahlen ist jede im Plenum anwesende
Fachschaft die mindestens eine teilnehmende Person hat.
Jede Fachschaft hat eine Stimme; wie sie abstimmt, ist innerhalb der
jeweiligen Fachschaft zu regeln.
Den Fachschaften ist Zeit zur Beratung zu gewähren.

\item Die normalen Personenwahlen sind wie folgt durchzuführen:
Die Kandidaten und Kandidatinnen stellen sich vor der Wahl kurz dem
Plenum vor.
Dem Plenum ist die Möglichkeit zu geben, unter Ausschluss der Kandidatinnen
und Kandidaten zu diskutieren.
Diese Diskussion wird nicht protokolliert.
Ein Kandidat oder eine Kandidatin gilt als gewählt, wenn er oder sie mehr
Ja-Stimmen als Nein-Stimmen, \emph{mindestens elf Ja-Stimmen}
erhält und die Wahl annimmt.
Enthaltungen sind möglich und wirken wie nicht oder ungültig abgegebene
Stimmen.
Sollten mehr Kandidatinnen und Kandidaten gewählt werden, als Posten zur
Verfügung stehen, werden sie nach Anzahl der Ja-Stimmen besetzt.

\item Im Anfangsplenum werden sechs Vertrauenspersonen gewählt. Zur Wahl
berechtigt sind alle anwesenden natürlichen Personen.

\item Die Wahl der Vertrauenspersonen erfolgt per Wahl durch
Zustimmung aus einem Pool von teilnehmenden Personen der ZaPF.
Bewerbungen hierfür müssen bis spätestens zu Beginn des Anfangsplenums
in schriftlicher Form an eine, bis spätestens zwei Wochen vor Beginn der
ZaPF durch die ausführende Fachschaft bekanntzugebende, Adresse erfolgen.

Der so bestimmten Gruppe muss anschließend mit absoluter Mehrheit vom
Plenum das Vertrauen ausgesprochen werden, damit sie als gewählt gelten.
Sind die ersten sechs Personen genannter Gruppe vom gleichen Geschlecht,
ersetzt die Person eines anderen Geschlechts mit den meisten Stimmen die
sechste Person in der Rangfolge.
Sollten sich nur Personen eines Geschlechts beworben haben, ist diese
Regelung irrelevant.

Bei weniger als sieben sich bewerbenden Menschen muss der kompletten Gruppe
das Vertrauen mit absoluter Mehrheit vom Plenum ausgesprochen werden,
damit sie als gewählt gelten.
Die Wahl durch Zustimmung entfällt hierbei.

Eine Personaldebatte findet nicht statt, die Kandidaten und Kandidatinnen
dürfen sich jedoch dem Plenum vorstellen.
Die Stimmverteilung wird nicht bekanntgegeben.
Die gewählten Vertrauenspersonen werden in alphabetischer Reihenfolge
vom Wahlausschuss veröffentlicht.

Darüber hinaus nominiert die austragende Fachschaft zwei Vertrauenspersonen
aus ihrer Fachschaft, diese müssen nicht vom Plenum bestätigt werden.

\item Wahl durch Zustimmung ist durch den folgenden Algorithmus definiert:

\begin{itemize}
\item Jede wahlberechtigte Person erhält einen Wahlzettel mit einer
Liste aller zur Wahl stehenden Personen.

\item Jeder zur Wahl stehenden Person kann eine Stimme gegeben werden.

\item Die Auszählung der Stimmen erfolgt in mehreren Durchgängen.

\item Im ersten Durchgang werden alle Stimmen ausgezählt und die Person
mit den meisten Stimmen kommt in die Gruppe der gewählten Personen.
Daraufhin werden alle Wahlzettel, die der ersten gewählten Person
eine Ja-Stimme gegeben haben, von den übrigen Wahlzetteln getrennt.

\item In den darauf folgenden Durchgängen wird immer die Person mit den
meisten Stimmen in den verbliebenen Wahlzetteln der Gruppe der gewählten
Personen hinzugefügt und ihre Wahlzettel von den übrigen Wahlzetteln
getrennt. Dies wird so lange wiederholt bis alle Plätze besetzt sind
oder keine Wahlzettel mehr übrig sind.

\item Sollten noch nicht alle Plätze in der Gruppe der gewählten Personen
besetzt sein obwohl keine Wahlzettel mehr verblieben sind, werden
die restlichen Plätze nach Anzahl der Stimmen in der ersten Runde
besetzt. Bei Gleichstand entscheidet das Los.
\end{itemize}

\item Abwahlen sind auch bei Abwesenheit der betroffenen Person möglich und
bedürfen einer Zweidrittelmehrheit. Der Antrag auf Abwahl ist bis spätestens
15 Uhr am Vortag der ausrichtenden Fachschaft anzukündigen.
Die betroffene Person ist jedoch nach Möglichkeit anzuhören.
\end{enumerate}


\subsection*{Anhang: Versionshistorie%
  \label{anhang-versionshistorie}%
}

Diese Geschäftsordnung wurde auf dem Abschlussplenum der Sommer-ZaPF 2005 in
Erlangen beschlossen.
Inhaltliche Änderungen wurden vorgenommen auf der:

\begin{itemize}
\item Sommer-ZaPF 2007 in Berlin,

\item Sommer-ZaPF 2008 in Konstanz,

\item Winter-ZaPF 2008 in Aachen,

\item Sommer-ZaPF 2009 in Göttingen,

\item Sommer-ZaPF 2010 in Frankfurt,

\item Sommer-ZaPF 2011 in Dresden

\item Sommer-ZaPF 2014 in Düsseldorf,

\item Winter-ZaPF 2014 in Bremen.

\item Sommer-ZaPF 2015 in Aachen,

\item Sommer-ZaPF 2016 in Konstanz,

\item Winter-ZaPF 2016 in Dresden,

\item Sommer-ZaPF 2017 in Berlin,

\item und auf der Winter-ZaPF 2017 in Siegen.
\end{itemize}


\subsection*{Anhang: Kommentare zur Geschäftsordnung und Begriffsklärung%
  \label{anhang-kommentare-zur-geschaftsordnung-und-begriffsklarung}%
}


\subsubsection*{Geschäftsordnungsanträge%
  \label{id1}%
}

Geschäftsordnungsanträge sind dazu gedacht, zu verhindern, dass eine Diskussion
sich ins Absurde zieht. Sie sind mit äußerster Vorsicht anzuwenden und sind
insbesondere als Korrektiv für eine Diskussion, die ihren roten Faden verloren
hat, zu benutzen.

Bei der Abstimmung über einen Geschäftsordnungsantrag sollte man vorher dreimal
darüber nachdenken, ob man ihm zustimmt, da Ende der Debatte auch Ende der Debatte
bedeutet.

Geschäftsordnungsanträge können als Mittel zu einer Schlammschlacht genutzt
werden, jedoch sollte bedacht werden, dass wir sachliche Diskussionen führen
wollen und auch einsehen sollten, wenn die Mehrheit einen Antrag nicht
unterstützt. Die GO kann nie so gefasst werden, dass sie weder von Teilnehmenden
des Plenums noch von der Redeleitung missbraucht werden kann. Für einen guten
Ablauf des Plenums sind wir auf das Wohlwollen aller angewiesen.

Um die GO-Anträge auf ihren einzigen Sinn, die Steuerung der Diskussion, zu
beschränken, wurden auf der ZaPF im Wintersemester 2014/2015 in Bremen die Liste
der GO-Anträge abgeschlossen und umfasst alle GO-Anträge die in der jüngeren
Vergangenheit benutzt wurden und die, die schon immer auf der Liste waren.
Dies umfasst unter anderem auch Verfahrensvorschläge,
wie z.B. die Entscheidung 2011 in Dresden eine ZaPF, um die sich mehrere
Fachschaften beworben hatten, per Stein-Schere-Papier zu vergeben.

Falls ein GO-Antrag nicht wie in der Liste benannt gestellt wird, versucht die
Redeleitung in Rücksprache einen inhaltsgleichen, korrekt gestellten Antrag zu
finden. Sollte die Redeleitung dabei einen Fehler macht, erinnert euch daran,
dass auch die Redeleitung nur aus Menschen besteht, die Fehler machen können und
weist sie darauf hin.

Abstimmungen ohne jegliche Gegenrede sollten nur mit äußerster Vorsicht
angenommen werden.

Formale Gegenrede bedeutet nur bekanntzugeben, dass man dagegen ist, inhaltliche
Gegenrede beinhaltet eine Begründung.


\subsubsection*{Beschlussfähigkeit bei zwanzig anwesenden Fachschaften%
  \label{beschlussfahigkeit-bei-zwanzig-anwesenden-fachschaften}%
}

Dies entspricht nach unserem Kenntnisstand etwa einem Viertel der Physikfachschaften.


\subsubsection*{Mindestanzahl von Ja-Stimmen bei Personenzahlen%
  \label{mindestanzahl-von-ja-stimmen-bei-personenzahlen}%
}

Das Minimum von elf Ja-Stimmen bewirkt, dass Kandidatinnen und Kandidaten
mindestens die absolute Mehrheit der zur Beschlussfähigkeit notwendigen Stimmen
erhalten muss.


\subsubsection*{Geeignete Form des Nachvollziehbarmachens%
  \label{geeignete-form-des-nachvollziehbarmachens}%
}

Es kann sehr schwer sein kleinste Änderungen in Texten nachzuvollziehen, es
erleichtert die Arbeit im Plenum deswegen erheblich, wenn Änderungen bestehender
Texte im einzelnen hervorgehoben sind. Dies kann z.B. durch ein Diff geschehen.


\subsubsection*{Resolutionen, Positionspapiere und Selbstverpflichtungen%
  \label{resolutionen-positionspapiere-und-selbstverpflichtungen}%
}

Resolutionen halten Positionen der ZaPF fest und werden vom StAPF an die im
Antrag angegebenen Adressaten verschickt.

Positionspapiere erfüllen den selben Zweck wie Resolutionen, aber haben keine
eigenen Adressaten und sollen im Bericht des StAPFes und auf der
Internetpräsenz der ZaPF in der Liste aller Resolutionen und Positionspapiere
veröffentlicht werden.

Selbstverpflichtungen sind ZaPF-interne Dokumente, die Aufträge an die Organe
der ZaPF, z.B. den StAPF, geben. Selbstverpflichtungen können insbesondere dafür
genutzt werden Arbeitsthesen eines Arbeitskreises festzuhalten, mit der
Intention auf einer folgenden ZaPF einen weiteren Beschluss zu fassen.


\subsubsection*{Konkurrierende Anträge%
  \label{konkurrierende-antrage}%
}

Konkurriende Anträge entfallen üblicherweise in eine von zwei Kategorien:

\begin{enumerate}
\item Verschiedene Änderungsanträge, die die selbe Textstelle ändern wollen.

\item Verschiede inhaltliche Beschlussfassungen zur selben Sache.
\end{enumerate}

\newpage