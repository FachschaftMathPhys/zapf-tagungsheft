% !TEX TS-program = pdflatex
% !TEX encoding = UTF-8 Unicode
% !TEX ROOT = main.tex

\newcommand*{\fett}[1]{\textbf{#1}}

\chapter{Glossar}

\begin{description}
	\item[AK] In \fett{A}rbeits\fett{k}reisen werden alle Themen der ZaPF bearbeitet, entweder in Form von offenen Diskussionen, inhaltlichem Austasuch, dem Abarbeiten konkreter Arbeitsaufträge oder dem Verfassen eines Antrags für das Plenum. Alle AKs werden protokolliert.
	\item[Akkreditierung] Die ZaPF kann Vertreter*innen in den studentischen Akkreditierungspool entsenden, die dann Studiengänge akkreditieren können. Daher beschäftigt sich die ZaPF fortlaufen mit dem Akkreditierungswesen in Deutschland.\footnote{Begriffserklärung: siehe zapf.wiki/Kategorie:Akkreditierung}
	\item[Alter Sack] Eigentlich wollen sie mittlerweile Alumni genannt werden, arbeiten oft noch inhaltlich mit, sind aber alt und sackig und ziehen sich langsam aus der Fachschaftsarbeit zurück
	\item[Austausch-AK] Habt ihr ein Problem? Fragt den Austausch-AK. In diesem AK werden Anliegen diskutiert, die keinen ganzen AK füllen würden und es werden Probleme einzelner Fachschaften diskutiert, denen gerade neue Impulse fehlen.
	\item[Bielefeld] Gibt es nicht
	\item[Bier-AK] Ein AK der keinen konkreten Zeitslot braucht, sondern über die ZaPF hinweg zwischen Tür und Angel oder aber während einem netten Abend bei einem Bier besprochen wird, nennt man Bier-AK. Es darf auch Sprudel konsumiert werden. 
	\item[CHE] Das \fett{C}entrum für \fett{H}ochschul\fett{E}ntwicklung ist verantwortlich für das CHE-Ranking bei dem alle drei Jahre Bachelorstudika zu ihrem Studium befragt werden. Die Ergebnisse werden dann über die ZEIT als Print- und Online-Version veröffentlicht, wobei einzelne Indikatoren Spitzen-, Mittel- und Schlussgruppen zugeordnet werden.
	\item[Ente] Wenn du glaubst, du siehst überall Enten, dann hast du nicht die Vogelgrippe. Vielmehr siehst du das schönste und tollste Tier der Welt, das zufällig auch das Maskottchen der ZaPF ist.
	\item[ewiges Frühstück] Zeit ist relativ und so gibt es rund um die Uhr Frühstück beim ewigen Frühstück.
	\item[Folge-AK] Beschäftigen wir uns über Jahre hinweg mit einem Thema resultiert das meist in Folge-AKs auf den nächsten ZaPFen. Das heißt auch, dass bereits Vorkenntnis in einem Folge-AK hilfreich ist, weil man bei der Diskussion nicht immer bei Null beginnen möchte. Deshalb lest vor einem Folge-AK die Protokolle der vorherigen ZaPFen.
	\item[GO] \fett{G}eschäfts\fett{O}rdnung
	\item[GO-Antrag] Ein Antrag zur Geschäftsordnung kann Debatten verkürzen
	\item[INF] \fett{I}m \fett{N}euenheimer \fett{F}eld ist der Straßenname aller Gebäude auf diesem Campus. Oft steht die Gebäudenummer groß auf dem Gebäude.
	\item[jDPG] Die \fett{j}ungen feshen Boys \& Girls der \fett{D}eutschen \fett{P}hysikalischen \fett{G}esellschaft schicken auch Vertretika auf die ZaPF. Zudem entsenden sie zwei Vertretika ins KommGrem.
	\item[KIP] \fett{K}irchhoff-\fett{I}nstitut für \fett{P}hysik (INF 226), Gebäude in dem die Plena und die meisten AKs stattfinden 
	\item[KIP2] Sieht aus wie das KIP und steht direkt neben dem KIP und ist über das Nordfoyer verbunden.
	\item[KFP] Die \fett{K}onferenz der \fett{F}achbereiche \fett{P}hysik ist die ZaPF der Dekane der Physik-Fakultäten.
	\item[KMK] Kultusministerkonferenz
	\item[KommGrem] Das \fett{Komm}unikations\fett{Grem}ium besteht aus 2 ZaPFika und 2 jDPGika die gegenseitig Positionen absprechen, da je ein vertretet zur KFP fährt. Die KommGremika leiten zudem die SachArbeit zum CHE (SACHE) und die Arbeit am Studienführer 
	\item[Kuschel-AK] Der Kuschel-AK beginnt mit dem Anfangsplenum. Er kann jederzeit durch Kuschelbeiträge bereichert werden.
	\item[LEuTe] Die \fett{L}ieblings-\fett{E}ngagierten in \fett{u}ngewählter \fett{T}askforc\fett{e} sind ZaPFika, die dermaßen übermotiviert sind, dass sie sich verpflichtet haben zwischen den ZaPFen inhaltliche Vor- oder Nacharbeit zu verrichten.
	\item[LRK] Landesrektorenkonferenz
	\item[MeTaFa] Die \fett{Me}ta-\fett{Ta}gung der \fett{Fa}chschaften ist der freiwillige Zusammenschluss der Bundesfachschaftentagungen. Da gehen meistens StAPFika für uns hin und diskutieren z.B. welche Resolutionen man gemeinsam vertreten möchte.
	\item[Plenum] 
	\item[Positionspapier] 
	\item[Postersession] 
	\item[Resolution] 
	\item[Reader] 
	\item[Satzung] 
	\item[StAPF] description
	\item[TO] 
	\item[TOP] description
	\item[TOPF] description
	\item[Vertrauensperon] description
	\item[ZaPF] description
	\item[ZaPF e.V.] description
	\item[ZaPF-Wiki] description
	\item[ZäPFchen] description
	\item[ZaPFika] description
\end{description}
