% !TEX TS-program = pdflatex
% !TEX encoding = UTF-8 Unicode
% !TEX ROOT = main.tex

\newcommand*{\fett}[1]{\textbf{#1}}

\chapter{Glossar}

\begin{description}
	\item[AK] In \fett{A}rbeits\fett{k}reisen werden alle Themen der ZaPF bearbeitet, entweder in Form von offenen Diskussionen, inhaltlichem Austasuch, dem Abarbeiten konkreter Arbeitsaufträge oder dem Verfassen eines Antrags für das Plenum. Alle AKs werden protokolliert.
	\item[Akkreditierung] Die ZaPF kann Vertreter*innen in den studentischen Akkreditierungspool entsenden, die dann Studiengänge akkreditieren können. Daher beschäftigt sich die ZaPF fortlaufen mit dem Akkreditierungswesen in Deutschland.\footnote{Begriffserklärung: siehe zapf.wiki/Kategorie:Akkreditierung}
	\item[Alter Sack] 
	\item[Austausch-AK] description
	\item[Bielefeld] Gibt es nicht
	\item[Bier-AK] 
	\item[CHE] Das \fett{C}entrum für \fett{H}ochschul\fett{E}ntwicklung 
	\item[Ente] description
	\item[Ente] description
	\item[ewiges Frühstück] description
	\item[Folge-AK] description
	\item[GO] description
	\item[GO-Antrag]
	\item[INF] \fett{I}m \fett{N}euenheimer \fett{F}eld ist der Straßenname aller Gebäude auf diesem Campus. Oft steht die Gebäudenummer groß auf dem Gebäude.
	\item[jDPG] 
	\item[KIP] \fett{K}irchhoff-\fett{I}nstitut für \fett{P}hysik (INF 226), Gebäude in dem die Plena und die meisten AKs stattfinden 
	\item[KIP2] description
	\item[KFP] 
	\item[KMK] Kultusministerkonferenz
	\item[KommGrem] Das \fett{Komm}unikations\fett{Grem}ium besteht aus 2 ZaPFika und 2 jDPGika die gegenseitig Positionen absprechen, da je ein vertretet zur KFP fährt. Die KommGremika leiten zudem die SachArbeit zum CHE (SACHE) und die Arbeit am Studienführer 
	\item[Kuschel-AK] Der Kuschel-AK beginnt mit dem Anfangsplenum. Er kann jederzeit durch Kuschelbeiträge bereichert werden.
	\item[LEuTe] Die \fett{L}ieblings-\fett{E}ngagierten in \fett{u}ngewählter \fett{T}askforc\fett{e} sind ZaPFika, die dermaßen übermotiviert sind, dass sie sich verpflichtet haben zwischen den ZaPFen inhaltliche Vor- oder Nacharbeit zu verrichten.
	\item[LRK] Landesrektorenkonferenz
	\item[MeTaFa] 
	\item[Plenum] 
	\item[Positionspapier] 
	\item[Postersession] 
	\item[Resolution] 
	\item[Reader] 
	\item[Satzung] 
	\item[StAPF] description
	\item[TO] 
	\item[TOP] description
	\item[TOPF] description
	\item[Vertrauensperon] description
	\item[ZaPF] description
	\item[ZaPF e.V.] description
	\item[ZaPF-Wiki] description
	\item[ZäPFchen] description
	\item[ZaPFika] description
\end{description}
