% !TEX TS-program = pdflatex
% !TEX encoding = UTF-8 Unicode
% !TEX ROOT = ../../main.tex

\newcommand*{\fett}[1]{\textbf{#1}}

\section{Glossar}

\begin{description}
	\item[AK] In \fett{A}rbeits\fett{k}reisen werden alle Themen der ZaPF bearbeitet, entweder in Form von offenen Diskussionen, inhaltlichem Austasuch, dem Abarbeiten konkreter Arbeitsaufträge oder dem Verfassen eines Antrags für das Plenum. Alle AKs werden protokolliert.
	\item[Akkreditierung] Die ZaPF kann Vertreter*innen in den studentischen Akkreditierungspool entsenden, die dann Studiengänge akkreditieren können. Daher beschäftigt sich die ZaPF fortlaufen mit dem Akkreditierungswesen in Deutschland.\footnote{Begriffserklärung: siehe zapf.wiki/Kategorie:Akkreditierung}
	\item[Alter Sack] Eigentlich wollen sie mittlerweile Alumni genannt werden, arbeiten oft noch inhaltlich mit, sind aber alt und sackig und ziehen sich langsam aus der Fachschaftsarbeit zurück
	\item[Austausch-AK] Habt ihr ein Problem? Fragt den Austausch-AK. In diesem AK werden Anliegen diskutiert, die keinen ganzen AK füllen würden und es werden Probleme einzelner Fachschaften diskutiert, denen gerade neue Impulse fehlen.
	\item[Bielefeld] Gibt es nicht
	\item[Bier-AK] Ein AK der keinen konkreten Zeitslot braucht, sondern über die ZaPF hinweg zwischen Tür und Angel oder aber während einem netten Abend bei einem Bier besprochen wird, nennt man Bier-AK. Es darf auch Sprudel konsumiert werden. 
	\item[CHE] Das \fett{C}entrum für \fett{H}ochschul\fett{E}ntwicklung ist verantwortlich für das CHE-Ranking bei dem alle drei Jahre Bachelorstudika zu ihrem Studium befragt werden. Die Ergebnisse werden dann über die ZEIT als Print- und Online-Version veröffentlicht, wobei einzelne Indikatoren Spitzen-, Mittel- und Schlussgruppen zugeordnet werden.
	\item[Ente] Wenn du glaubst, du siehst überall Enten, dann hast du nicht die Vogelgrippe. Vielmehr siehst du das schönste und tollste Tier der Welt, das zufällig auch das Maskottchen der ZaPF ist.
	\item[ewiges Frühstück] Zeit ist relativ und so gibt es rund um die Uhr Frühstück beim ewigen Frühstück.
	\item[Folge-AK] Beschäftigen wir uns über Jahre hinweg mit einem Thema resultiert das meist in Folge-AKs auf den nächsten ZaPFen. Das heißt auch, dass bereits Vorkenntnis in einem Folge-AK hilfreich ist, weil man bei der Diskussion nicht immer bei Null beginnen möchte. Deshalb lest vor einem Folge-AK die Protokolle der vorherigen ZaPFen.
	\item[GO] \fett{G}eschäfts\fett{O}rdnung
	\item[GO-Antrag] Ein Antrag zur Geschäftsordnung kann Debatten verkürzen
	\item[HUMBUG] \fett{H}eutiges \fett{U}nd Generelles zu \fett{M}aster, \fett{B}achelor \fett{U}nd \fett{G}enerellem
	\item[INF] \fett{I}m \fett{N}euenheimer \fett{F}eld ist der Straßenname aller Gebäude auf diesem Campus. Oft steht die Gebäudenummer groß auf dem Gebäude.
	\item[jDPG] Die \fett{j}ungen feshen Boys \& Girls der \fett{D}eutschen \fett{P}hysikalischen \fett{G}esellschaft schicken auch Vertretika auf die ZaPF. Zudem entsenden sie zwei Vertretika ins KommGrem.
	\item[KIP] \fett{K}irchhoff-\fett{I}nstitut für \fett{P}hysik (INF 226), Gebäude in dem die Plena und die meisten AKs stattfinden 
	\item[KIP2] Sieht aus wie das KIP und steht direkt neben dem KIP und ist über das Nordfoyer verbunden.
	\item[KFP] Die \fett{K}onferenz der \fett{F}achbereiche \fett{P}hysik ist die ZaPF der Dekane der Physik-Fakultäten.
	\item[KMK] Kultusministerkonferenz
	\item[KommGrem] Das \fett{Komm}unikations\fett{Grem}ium besteht aus 2 ZaPFika und 2 jDPGika die gegenseitig Positionen absprechen, da je ein vertretet zur KFP fährt. Die KommGremika leiten zudem die SachArbeit zum CHE (SACHE) und die Arbeit am Studienführer
	\item[Konsumikon] Im Mathemikon befinden sich im hinteren Bereich zwei Supermärkte, zwei Bäckereien und ein Drogeriemarkt, falls manche Bedürfnisse vom ewigen Frühstück nicht befriedigt werden können.
	\item[Kuschel-AK] Der Kuschel-AK beginnt mit dem Anfangsplenum. Er kann jederzeit durch Kuschelbeiträge bereichert werden.
	\item[LEuTe] Die \fett{L}ieblings-\fett{E}ngagierten in \fett{u}ngewählter \fett{T}askforc\fett{e} sind ZaPFika, die dermaßen übermotiviert sind, dass sie sich verpflichtet haben zwischen den ZaPFen inhaltliche Vor- oder Nacharbeit zu verrichten.
	\item[LRK] Landesrektorenkonferenz
	\item[Mathematikon] Das neue Mathegebäude auf dem Campus (INF 205) in dem sich unser regulärer FS-Raum befindet.
	\item[MeTaFa] Die \fett{Me}ta-\fett{Ta}gung der \fett{Fa}chschaften ist der freiwillige Zusammenschluss der Bundesfachschaftentagungen. Da gehen meistens StAPFika für uns hin und diskutieren z.B. welche Resolutionen man gemeinsam vertreten möchte.
	\item[Plenum] Das Plenum ist in erster Linie das beschlussfassende Organ der ZaPF. Zu Beginn jeder ZaPF gibt es ein Anfangsplenum, das meist nur zur Klärung des Ablaufs und Vorstellung der AKs dient. Es können aber auch bereits vorliegende Satzungsänderungsvorschläge diskutiert und beschlossen werden. Die Ergebnisse aus den AKs werden dann im Abschlussplenum abschließend diskutiert und wenn gewünscht auch konkrete Beschlüsse, wie Resolutionen oder Positionspapiere beschlossen. Für Personenentscheidungen ist ebenfalls das Plenum zuständig.
	\item[Positionspapier] Veröffentlichung der ZaPF zu einem konkreten Sachinhalt ohne Adressaten.  
	\item[Postersession] Während des Zwischenplenums werden die Ergebnisse, besonders erste Entwürfe von Beschlussfassungen, der bereits stattgefundenen AKs besprochen. Herrscht noch Redebedarf, kann dies geäußert werden. Dies führt dazu, dass nach dem Zwischenplenum im kleinen Kreis Formulierungen umgeschrieben, Missverständnisse ausgeräumt und kritische Punkte diskutiert werden können. 
	\item[Resolution] Schriftliche Meinungsäußerung der ZaPF mit konkreten Adressaten. In der Regel werden diese AKs vorbereitet und vom Plenum beschlossen.
	\item[Reader] Nach der ZaPF werden die Ergebnisse in Form von Protokollen und Beschlüssen in einem Reader zusammengestellt. Dieser wird auf der nächsten ZaPF verteilt und online gestellt. 
	\item[SACHE] \fett{S}ach\fett{A}rbeit am \fett{CHE}
	\item[Satzung] Die Satzung der ZaPF regelt diese, ihre Organe und ihr Tun. Siehe Abschnitt \nameref{satzung}
	\item[StAPF] Der \fett{St}ändige \fett{A}usschuss der \fett{P}hysik-\fett{F}achschaften vertritt die ZaPF in der Öffentlichkeit und ist die zentrale Ansprech-/ Koordinationsstelle zwischen den ZaPFen.
	\item[TO] In der \fett{T}ages-\fett{O}rdnung wird festgelegt, was in welcher Reihenfolge besprochen wird.
	\item[TOP] Ein \fett{T}ags-\fett{O}rdnungs-\fett{P}unkt ist Element der TO
	\item[TOPF] Der \fett{T}echnische \fett{O}rganisationsausschuss aller \fett{P}hysik-\fett{F}achschaften kümmert sich um die IT der ZaPF. Koordiniert wird deren Arbeit durch die gewählten DECkEL (Dokumentations-, Einrichtungs-, und Clusterfuckkoordinatorika für EDV-Lösungen) mithilfe engagierter HENkeL (Helfer*innen mit EDV- und Netzwerkkompetenzen für ergebnisorientierte Lösungen). 
	\item[Vertrauensperon] Auf der ZaPF gibt es ein Team von Vertrauenspersonen, die Hilfesuchenden als Anlaufsstelle dienen. Das Team besteht aus zwei ZaPFika aus der Orga und bis zu sechs externen ZaPFika.
	\item[ZaPF] \fett{Z}usammenkunft \fett{a}ller \fett{P}hysik-\fett{F}achschaften; hier bist du gerade
	\item[ZaPF e.V.] Das ist ein eng an die ZaPF angelehnter Verein, der v.a. die Finanzmittel zur Ausrichtung einer ZaPF verwaltet.
	\item[ZaPF-Wiki] Auf https://zapf.wiki/ findet man die Protokolle der vergangenen ZaPFen, die GO, die Satzung und auch sonst allerhand wissenswertes. 
	\item[ZäPFchen] Wer zum ersten Mal auf der ZaPF ist, gilt als ZäPFchen. Für ZäPFchen gibt es nach dem Anfangsplenum einen ZäPFchen-AK bei dem ein paar grundlegende Abläufe erklärt werden.
	\item[ZaPFika] \#(Teilnehmika einer ZaPF) > 2
\end{description}