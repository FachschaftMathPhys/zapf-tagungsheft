% !TEX TS-program = pdflatex
% !TEX encoding = UTF-8 Unicode
% !TEX ROOT = main.tex

\section[Heidelberg und die Univsersität ]{Heidelberg und die Univsersität - eine kleine Geschichte}

Es ist gar nicht einmal so einfach einen markanten Anfangspunkt für die Geschichte Heidelbergs zu finden, aber da Heidelberg immer sehr gerne darauf besteht, in allem möglichen das älteste zu haben, fangen wir ganz weit vorne an: Mit dem \textit{Homo Heidelbergensis}.

Dieser Vorfahre des \textit{Homo sapiens} stammt ursprünglich aus Afrika, ist vor ca. 600 000 nach Mitteleuropa gezogen und hat sich hier zu dem entwickelt, was wir als \textit{Neanderthaler} kennen. Andererseits wurde ein Unterkiefer dieser Art erstmals 1907 in der Umgebung von Heidelberg gefunden und ist dem Heidelberger Paläontologen Otto Schoetensack zugetragen worden, was offensichtlich ein \textit{exzellenter} Grund ist, diese Art nach Heidelberg zu benennen.

Obwohl Heidelberg somit die ältesten zumindest menschenähnlichen Überreste Mitteleuropas aufzuweisen hat, muss man etwas Geduld mitbringen, um es zum nächsten Superlativ der Stadtgeschichte zu schaffen. Ca. 500 v. Chr. gründeten die Kelten eine Siedlung auf einem der Heidelberger Stadtberge, allerdings wurde diese schon nach 200 Jahren aus unbekannten Gründen wieder aufgegeben. Zwischen 40 und 260 n. Chr. unterhielten die Römer hier ein Kastell zum Schutz einer Neckarbrücke, in dessen Umfeld sich eine so kleine Ortschaft bildete, dass nicht einmal ihr lateinischer Name überliefert ist.

Im frühen Mittelalter wurden im vorderen Neckartal einige Klöster gegründet, in deren Unterlagen sich die erste schriftliche Erwähnung von \textit{Haydelberg} im Jahr 1196 findet. Mitte des 14. Jhdts. wurde Heidelberg zur Residenzstadt der Fürsten der Kurpfalz und 1386 wurde von Ruprecht I. die Universität Heidelberg gegründet.

Die Universität Heidelberg war nach Prag und Wien die drittälteste Universität des Heiligen Römischen Reiches und ist somit die älteste Universität auf dem Gebiet der Bundesrepublik. Obwohl in den Anfangsjahren nur wenige Hundert Studenten und auch nur in Philosophie, Theologie, Jura und Medizin unterrichtet wurden, hatte die Universität nicht genügend eigene Räume, sodass die Vorlesungen meist in den Klöstern der Stadt stattfanden. Das hat sich 1390 geändert als der Universität Häuser und Besitz der aus Heidelberg vertriebenen Juden vermacht wurden.

Aufgrund der günstigen geografischen Lage und der politischen Macht der Kurfürsten erlebte Heidelberg im 15. und 16. Jhdt. eine Blütezeit: Die Stadt wurde mehrmals bis auf die Größe der heutigen Altstadt erweitert, das berühmte Heidelberger Schloss wurde errichtet und die Universität wurde zu einem der akademischen Zentren der Reformation. In dieser Zeit wurde auch die \textit{Bibliotheca palatina} aufgebaut, die erste Universitätsbibliothek Deutschlands und die bedeutendste Bibliothek Mitteleuropas in der Renaissance.

Das alles kam Anfang des 17. Jhdt. zu einem jähen Ende: 1619 wurde Friedrich dem V., dem protestantischen Kurfürsten, die böhmische Krone angetragen und er nahm diese gegen den Willen des katholischen Kaisers an. Es kam zum Krieg zwischen dem \textit{Winterkönig} Friedrich V. und den kaiserlichen Habsburgern, dem Anfang dessen, was heute als der Dreißigjährige Krieg bekannt ist.

So kam es, dass 1622 kaiserliche Truppen Heidelberg eroberten und das Umland verwüsteten. Die Universität musste ihren Betrieb einstellen und der Papst beanspruchte die Bibliotheca palatina als Kriegsbeute. Im Westfälischen Frieden wurde die Kurpfalz samt Heidelberg wiederhergestellt, die Bibliotheca palatina verblieb allerdings im Vatikan, wo sich noch heute ein Großteil der historischen Bestände befindet.

Auch nach dem Ende des Dreißigjährigen Krieges wurde Heidelberg 1688 und 1693 von den Franzosen besetzt, die bei ihrem Abzug das Schloss systematisch sprengten. Wegen fehlendem Interesse und klammer Kassenlage wurde das Schloss nie vollständig restauriert und diente der Heidelberger Stadtbevölkerung als informeller Steinbruch, bis sich einige Liebhaber ab Anfang des 19. Jhdts. für den Erhalt der malerischen Schlossruine einsetzten. Mit dem Verfall des Schlosses und der Verlegung der kurfürstlichen Residenz nach Mannheim ging auch der politische Bedeutungsverlust Heidelbergs einher.

Im Jahr 1803 wurde Heidelberg Teil des Großherzogtum Badens und die Universität wurde eine staatlich finanzierte Lehranstalt. In der Folge kam frischer Wind in das geistige Leben Heidelbers, so fällt z.B. das Schaffen von Bunsen, Helmholtz und Kirchhoff wie auch die Heidelberger Romantik in diese Phase.

In den 1930er und 1940er Jahren wiederum zeigte sich die Universität und insbesondere die Physik in Heidelberg nicht von ihrer besten Seite, als sie mit besonderem Eifer den Weg zur sog. \textit{nationalsozialistischen Universität} beschritt. Baulich verbleibt Heidelberg aus der Zeit des Nationalssozialismus vor allem die \textit{Heidelberger Thingstätte}, eine nach antikem Vorbild entworfene Freilichtbühne; aber schon während der Zeit des Nationalsozialismus verloren die Erbauer selbst größtenteils das Interesse an ihrer Propagandabühne und heutzutage wird sie vor allem in der Walpurgisnacht vom entgegengesetzten Ende des politischen Spektrums genutzt.

Von Kriegsschäden größtenteils verschont wurde Heidelberg in der Nachkriegszeit eine zentrale Garnisonsstadt der US-Armee, wovon heute noch hektarweise verlassene Kasernen zeugen. Die Universität wurde 1946 wiedereröffnet und zu dieser Zeit wurde auch das \textit{Collegium Academicum}, ein selbstverwaltetes studentisches Kollegienhaus, etabliert. Ziel der US-Behörenden war es unter anderen die neue Generation Studierender Demokratie auch in ihrem Wohnalltag nahezubringen und einen Gegenentwurf zu den direkt nach dem Krieg noch verbotenen Burschenschaften anzubieten.

Im Herbst 1977 wurde der Studierendenschaft in Heidelberg Verbindungen zur RAF vorgeworfen (was in Teilen auch zutraf) und unter diesem Vorwand wurden die Verfassten Studierendenschaften in Baden-Württemberg und Bayern verboten. Im Frühjahr 1978 wurde auch das Collegium Academicum aufgelöst und von einigen Hundertschaften der Bereitschaftspolizei gestürmt. In das Gebäude zog in der Folge die Universitätsverwaltung.
