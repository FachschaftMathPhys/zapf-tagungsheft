\section{Internet \& Strom}
  % Bild einer Schweinenase <- dies ist keine Steckdose
  \paragraph{WLAN}
  Das Internet ... Jeder braucht es und in einer der europäischen Digitalen Städte gibt es natürlich an den meisten Ecken und erst Recht an unseren Uni
  öffentliches WLAN. Wenn ihr schon im Netz ``eduroam'' registriert seid, weil es das an eurer Hochschule auch gibt - super!
  Ihr könnt euch zurücklegen und braucht gar nichts weiter zu machen. \\
  Wenn ihr noch nicht direkt verbunden werdet, gibt es hier auch das öffentliche Netz ``Heidelberg4You''.
  Es ist überall dort verfügbar, wo die ZaPFika neben euch im ``eduroam'' sind und sogar noch an weiteren Stellen in Heidelberg.
  Die Anmeldung dort funktioniert recht einfach über ein Pop-Up Window, in dem ihr einfach die Lizenzbedingungen \textit{lesen} und bestätigen müsst.
  Danach habt ihr unbegrenzten Zugriff, um noch produktiver für die ZaPF zu arbeiten oder euch von der NSA abhören zu lassen. \\

  \paragraph{Drucken}
  Solltet ihr mal lebenswichtige Papiere auszudrucken haben, geht das im Tagungsbüro. Dort gibt es jede Menge Computer, die eure Dateien
  öffnen können sollten und sich auch mit den Druckern dort verbinden können.
  Natürlich ist das ganze nicht dazu gedacht eure ganzen Fotos vom letzten Urlaub auszudrucken (die Qualität ist eh nicht die beste),
  sondern lediglich dazu gedacht, die Resolutionen in Papierform zu haben, den AK Leitern das nötige Material zu beschaffen und solche
  Tagungsdinge eben, die anfallen.

  \paragraph{Strom}
  Ein Leben ohne Strom? Undenkbar! 
  Auf den Zimmern der Jugendherbege wird es genug Steckdosen geben, um eure klugen Handys über Nacht

  \todo{funktioniert das Heidelberg4You wie beschrieben?}
  \todo{Gastzugänge im eduroam}
  \todo{passende Fotos raussuchen}
