% !TEX TS-program = pdflatex
% !TEX encoding = UTF-8 Unicode
% !TEX ROOT = main.tex

\section{Internet \& Strom}
  % Bild einer Schweinenase <- dies ist keine Steckdose
\paragraph{WLAN}
Das Internet \dots Jeder braucht es und bei uns gibt's das. Praktischerweise habt ihr an den meisten Ecken und erst Recht an unserer Uni öffentliches WLAN. Wenn ihr schon im Netz \glqq eduroam\grqq registriert seid, weil es das an eurer Hochschule auch gibt - super! Ihr könnt euch zurücklehnen und braucht gar nichts weiter zu machen.

Wenn ihr noch nicht direkt verbunden werdet oder schlichtweg keinen Zugang habt, gibt es im Tagungsbüro Gastaccounts für euch. Außerdem gibt es auch das öffentliche Netz \glqq Heidelberg4You\grqq. Es ist überall dort verfügbar, wo die ZaPFika neben euch im \glqq eduroam\grqq sind und sogar noch an weiteren Stellen in Heidelberg. Die Anmeldung dort funktioniert recht einfach über ein Pop-Up Window, in dem ihr einfach die Lizenzbedingungen \textit{lesen} und bestätigen müsst. Danach habt ihr unbegrenzten Internetzugang, um noch produktiver für die ZaPF zu arbeiten oder euch von der NSA abhören zu lassen. Um letzteres zu verhindern, bietet sich auch die Verwendung eines Virtual Private Network (VPN) an.

\paragraph{Drucken}
Solltet ihr mal lebenswichtige Papiere auszudrucken haben, geht das im Tagungsbüro. Dort gibt es jede Menge Computer, die eure Dateien öffnen können sollten und sich auch mit den Druckern dort verbinden können. Natürlich ist das Ganze nicht dazu gedacht, all eure Fotos vom letzten Urlaub auszudrucken (die Qualität ist eh nicht die beste), sondern lediglich dazu, die Resolutionen in Papierform vorliegen zu haben, den AK Leitika das nötige Material zu beschaffen und solche Tagungsdinge eben, die anfallen.

\paragraph{Strom}
Ein Leben ohne Strom? Undenkbar! Auf den Zimmern der Jugendherbege wird es genug Steckdosen geben, um eure klugen Handys über Nacht zu laden und allen möglichen anderen Shice damit zu machen. In den AK Räumen gibt es auf der Fensterseite genug Steckdosen, wenn eurem Laptop doch tagsüber mal der Saft ausgeht. In dem Plenums Hörsaal wird ebenfalls für ausreichend Anschlüsse gesorgt, damit es währenddessen schön spannend bleibt. Da der ein oder andere von euch aus Versehen die Verlängerungskabel vergessen haben wird, liegen ein paar im Tagungsbüro aus.