% !TEX TS-program = pdflatex
% !TEX encoding = UTF-8 Unicode
% !TEX ROOT = main.tex

\section{Kommunikationskanäle}

\subsection{Website}
\url{www.zapfinhd.de}\\

\noindent Auf der Website sollten alle wichtigen Informationen zur Konferenz zu finden sein, teilweise aktueller als hier im Tagungsheft

\subsection{Wiki}
\url{www.zapf.wiki}\\

\noindent Im Wiki werden die Arbeitskreise angekündigt, Protokolle angelegt sowie Resolutionen und Positionspapiere veröffentlicht.

\subsection{Pad}
\url{URL}\\ \todo{Die URL des Pad steht noch nicht fest}

\noindent In Pads können Protokolle verfasst und Entwürfe für Resos erstellt werden.

\subsection{Mailingliste}
\url{Mailinglisten Email}\\ \todo{gibt es eine Mailingliste?}

\noindent Über die Mailingliste können alle Teilnehmika erreicht werden.

\subsection{Telegram Broadcast}
\url{Mailinglisten Email} \\

\noindent Hierrüber werden während der ZaPF wichtige Infos gesendet.

\subsection{Engelsystem}
\url{www.engel.zapfinhd.de}\\

\noindent Damit die ZaPF funktioniert, werden viele Engel benötigt. Wenn du auch mithelfen möchtest, kannst du dies über das Engelsystem tun.