% !TEX TS-program = pdflatex
% !TEX encoding = UTF-8 Unicode
% !TEX ROOT = main.tex

\section{Kommunikationskanäle}

\subsection{Website \hfill \url{www.zapfinhd.de}}

\noindent Auf der Website sollten alle wichtigen Informationen zur ZaPF zu finden sein, teilweise aktueller als hier im Tagungsheft

\subsection{Wiki \hfill \url{www.zapf.wiki}}

\noindent Im Wiki werden die Arbeitskreise angekündigt, Protokolle angelegt, sowie Resolutionen und Positionspapiere veröffentlicht.

\subsection{Mailinglisten}

\begin{itemize}%\todo{gibt es eine Mailingliste?}
\item[\faEnvelope] \url{zapfika@mathphys.stura.uni-heidelberg.de} Über diese Mailingliste können alle ZaPFika erreicht werden. Wenn ihr etwas habt, was über diese Liste versendet werden soll, wendet euch ans Tagungsbüro.

\item[\faEnvelope] \url{resos@zapf.in} Entwürfe für Resolutionen, Positionspapiere und andere Dokumente, über die im Plenum abgestimmt werden soll, sollten an diese Mailadresse gesendet werden. Sie kommen dann der Redeleitung zu, die sie in die Plenen einarbeitet.

\item[\faEnvelope] \url{plenum@zapf.in}  Wenn ihr eine Anfrage oder einen Input für das Plenum habt, die keine Resolutionen sind, könnt ihr über diese Mailingliste die Redeleitung, Plenumstechnik und die Orga erreichen.

\end{itemize}

\subsection{Telegram Broadcast \hfill \url{Telegram Broadcast}} %todo

\noindent Hierrüber werden während der ZaPF wichtige Infos gesendet. Du kannst hier zwar keine Fragen stellen, aber vielleicht wurde deine Frage ja schon einmal beantwortet.

\subsection{Engelsystem \hfill \url{www.engel.zapfinhd.de}}

\noindent Damit die ZaPF funktioniert, werden viele Helfika benötigt. Wenn du auch mithelfen möchtest, kannst du dich hier in Schichten eintragen.