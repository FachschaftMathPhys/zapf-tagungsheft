% !TEX TS-program = pdflatex
% !TEX encoding = UTF-8 Unicode
% !TEX ROOT = main.tex

\section{Kommunikationskanäle}

\subsection{Website}
\url{www.zapfinhd.de}\\

\noindent Auf der Website sollten alle wichtigen Informationen zur Konferenz zu finden sein, teilweise aktueller als hier im Tagungsheft

\subsection{Wiki}
\url{www.zapf.wiki}\\

\noindent Im Wiki werden die Arbeitskreise angekündigt, Protokolle angelegt sowie Resolutionen und Positionspapiere veröffentlicht.

\subsection{Pad}
\url{URL}\\ \todo{Die URL des Pad steht noch nicht fest}

\noindent In Pads können Protokolle verfasst und Entwürfe für Resos erstellt werden.

\subsection{Mailinglisten}
\url{Mailinglisten Email}\\ \todo{gibt es eine Mailingliste?}

\noindent Über diese Mailingliste können alle ZaPFika erreicht werden. Wenn ihr etwas habt, was über diese Liste versendet werden soll, wendet euch ans Tagungsbüro.

\url{resos@zapf.in}

Entwürfe für Resolutionen, Positionspapiere und andere Dokumente, über die im Plenum abgestimmt werden soll, sollten an diese Mailadresse gesendet werden. Sie kommen dann der Redeleitung zu, die sie in die Plenen einarbeitet.

\url{plenum@zapf.in}

\noindent Wenn ihr eine Anfrage oder einen Input für das Plenum habt, der keine Resolution ist, könnt ihr über diese Mailingliste die Redeleitung, Plenumstechnik und die Orga erreichen.

\subsection{Telegram Broadcast}
\url{Mailinglisten Email} \\

\noindent Hierrüber werden während der ZaPF wichtige Infos gesendet. Du kannst hier zwar keine Fragen stellen, aber vielleicht wurde deine Frage ja schon einmal beantwortet.

\subsection{Engelsystem}
\url{www.engel.zapfinhd.de}\\

\noindent Damit die ZaPF funktioniert, werden viele Helfika benötigt. Wenn du auch mithelfen möchtest, kannst du dich hier in Schichten eintragen.