% !TEX TS-program = pdflatex
% !TEX encoding = UTF-8 Unicode
% !TEX ROOT = main.tex

\section{Schlafen \& Duschen}
Ihr habt das große Glück, dass die Heidelberger Stadtverwaltung ihre öffentlichen Räumlichkeiten nicht an uns vermieten will. Das heißt konkret, dass die Unterbringung in existierenden Schlafstätten stattfindet, also ihr in der örtlichen Jugendherbege und in Notunterkünften des Studierendenwerks untergebracht seid. Natürlich ergeben sich dann auch viele Vereinfachungen für euren Alltag! \\
\begin{itemize}
  \item Es gibt richtige Matratzen für euren Schönheitsschlaf
  \item Eure Schlafsachen und das Gepäck könnt ihr bis zur Abreise auf den Zimmern lassen
  \item Die Zimmer können abgeschlossen werden
  \item Nachts gibt es genug Steckdosen
  \item Duschen für alle!
\end{itemize}

\todo[inline]{Wie ist die Unterbringung in den Notunterkünften?}
Ihr schwer arbeitenden ZaPFika habt euch diesen Luxus hart erarbeitet und wir sind uns mit unseren Sponsoren einig, dass ihr es auch verdient! \\

Zum Vergolden der Toiletten und Armaturen hatten wir leider nicht genug Zeit, aber es gibt auf jeden Fall genug Stille Orte in Reichweite der Tagungsräume und auf den Zimmern. Auf dem Gebäuderaumplan \todo[inline]{ref. zum Gebäuderaumplan}
sind mehrere Alternativen angegeben. Wenn du dann mal bemerken solltest, dass die Dinge des täglichen Bedarfs zur Neige gehen, melde dich am besten kurz im Tagungsbüro. \todo[inline]{ref. zur Tagungsbüro Telefon Nummer} Dann können wir entsprechend reagieren. Das Nachfüllen passiert leider noch nicht automatisiert bei uns - das Robotiklabor ist erst in der Alpha-Phase.
