% !TEX TS-program = pdflatex
% !TEX encoding = UTF-8 Unicode
% !TEX ROOT = main.tex

\section{Anlaufstellen}

\subsection{Anmeldung}
\faMapPin \quad Foyer der Jugendherberge Heidelberg\\
\faClockO \quad \\ \todo{Öffnungszeiten der Anmeldung}

\noindent Bei der Anmeldung bekommt man seine Tagungstasche und einen Badge.

\subsection{Tagungsbüro}
\faPhone \quad +49 6221 54 ???\\ \todo{Nummer vom PI}
\faMapPin \quad INF 226 (Physikalisches Institut), Raum 00.210\\
\faClockO \quad \\ \todo{Öffnungszeiten des Tagungsbüros einfügen}

\noindent Das Tagungsbüro sollte immer die erste Anlaufstelle sein. Dort wird euch kompetent weitergeholfen, egal was ihr wollt.

\subsection{Orga-Büro}
\faPhone \quad +49 6221 54 ???\\ \todo{Nummer vom PI}
\faMapPin \quad INF 226 (Physikalisches Institut), Raum 00.210\\
\faClockO \quad \\ \todo{Nicht durchgängig besetzt}

\noindent Bei wichtigen Dingen kann man im Orga Büro vorbei kommen.

\subsection{Vertrauenspersonen}
\faPhone \quad +49 ???? ????\\ \todo{Nummer von Vertrauenspersonen}
\faUsers \quad \\ \todo{Namen der Vertrauenspersonen}

\noindent \todo[inline]{Text zu Vertrauenspersonen}




