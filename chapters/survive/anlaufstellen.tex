% !TEX TS-program = pdflatex
% !TEX encoding = UTF-8 Unicode
% !TEX ROOT = ../../main.tex

\section{Anlaufstellen}

\subsection{Tagungsbüro}
\begin{tabbing}
\faPhone \quad \quad \= +49 1577 10 71 821 \quad \quad \faClockO \= \quad \quad  06:00 bis 02:00 \\ %\todo{Nummer vom Tagungsbüro}
\faMapPin \> INF 227 (Kirchhoff-Institut für Physik), Raum 1.403 % Tagungsbüro ist wo anders, nicht =Orga-Büro
\end{tabbing}

\noindent Das Tagungsbüro sollte immer die erste Anlaufstelle sein. Dort wird euch kompetent weitergeholfen, egal was ihr wissen wollt oder braucht.

\subsection{Orga-Büro}
\begin{tabbing}
\faPhone \quad \quad \= +49 6221 54 19555 \quad \quad \faClockO \= \quad \quad  Semper apertus - immer offen\\
\faMapPin \> INF 226 (Physikalisches Institut), Raum 00.210
\end{tabbing}

\noindent Im Regelfall solltet ihr hier nicht vorbeikommen müssen. Wenn das Tagungsbüro nachts zu ist, übernimmt die Zentrale allerdings die Rolle des Tagungsbüros. Geht mit normalen Nachfragen bitte zuerst zum Tagungsbüro. % Grade nachts, wenn das Tagungsbüro nicht besetzt ist.

%\subsection{Anmeldung}
%\begin{tabbing}
%\faMapPin \quad \quad \= Foyer der Jugendherberge Heidelberg\\ %nein, noch unklar
%\faClockO \> 14:00 bis 18:00, ab 18:00 im Tagungsbüro % bis Plenumsbeginn 18:00, danach Tagungsbüro
%\end{tabbing}

%\noindent Bei der Anmeldung bekommt man seine Tagungstasche und einen Badge\footnote{Aber wenn du dieses Heft in Händen hältst, bist du schon bei der Anmeldung gewesen.}. % Außerdem muss hier noch der Teilnehmerbeitrag bezahlt werden, falls noch nicht geschehen.

\subsection{Vertrauenspersonen}
\faPhone \quad \quad +49 ???? ???? (Thomi) \quad \quad +49 ???? ???? (Irina)\\%\todo{Nummer von Vertrauenspersonen} % Irina? Thomi?
%\faUsers \> Thomi, Irina \\ %\todo{Namen der Vertrauenspersonen}

\noindent Die Vertauenspersonen wurden auf der Winter-ZaPF 2013 in Wien eingeführt, um als Ansprech- und GesprächspartnerInnen in Fällen von Ausgrenzung, Diskriminierung und Belästigung zu fungieren. Sie sollen eine entsprechende Situation von außen betrachten und einschätzen können und haben sich die zur Diskretion gegenüber den Hilfesuchenden verpflichtet.
%\todo[inline]{Text zu Vertrauenspersonen}
