% !TEX TS-program = pdflatex
% !TEX encoding = UTF-8 Unicode
% !TEX ROOT = main.tex

\section{Anlaufstellen}

\subsection{Tagungsbüro}
\faPhone \quad +49 6221 54 ???\\ %\todo{Nummer vom Tagungsbüro}
\faMapPin \quad INF 226 (Physikalisches Institut), Raum 00.210\\ % Tagungsbüro ist wo anders, nicht =Orga-Büro
\faClockO \quad 06:00 bis 02:00 % 6:00 - 2:00

\noindent Das Tagungsbüro sollte immer die erste Anlaufstelle sein. Dort wird euch kompetent weitergeholfen, egal was ihr wissen wollt oder braucht.

\subsection{Orga-Büro}
\faPhone \quad +49 6221 54 19555\\
\faMapPin \quad INF 226 (Physikalisches Institut), Raum 00.210\\
\faClockO \quad Semper apertus - immer offen

\noindent Im Regelfall solltet ihr hier nicht vorbeikommen müssen. Wenn das Tagungsbüro nachts zu ist, übernimmt die Zentrale allerdings die Rolle des Tagungsbüros. Geht mit normalen Nachfragen bitte zuerst zum Tagungsbüro. % Grade nachts, wenn das Tagungsbüro nicht besetzt ist.

\subsection{Vertrauenspersonen}
\faPhone \quad +49 ???? ????\\ %\todo{Nummer von Vertrauenspersonen} % Irina? Thomi?
\faUsers \quad \\ %\todo{Namen der Vertrauenspersonen}

\noindent %\todo[inline]{Text zu Vertrauenspersonen}

\subsection{Anmeldung}
\faMapPin \quad Foyer der Jugendherberge Heidelberg\\ %nein, noch unklar
\faClockO \quad 14:00 bis 18:00, ab 18:00 im Tagungsbüro % bis Plenumsbeginn 18:00, danach Tagungsbüro

\noindent Bei der Anmeldung bekommt man seine Tagungstasche und einen Badge. Aber wenn du dieses Heft in Händen hältst, bist du wahrscheinlich auch schon bei der Anmeldung gewesen. % Außerdem muss hier noch der Teilnehmerbeitrag bezahlt werden, falls noch nicht geschehen.

