% !TEX TS-program = pdflatex
% !TEX encoding = UTF-8 Unicode
% !TEX ROOT = main.tex

\section{Essen}
Heidelberg will sich für euch natürlich nur von seiner besten Seite zeigen. Deshalb haben wir keine Kosten und Mühen gescheut,  um euch ein Essen der Extraklasse präsentieren zu können. Die logistische Meisterleistung für $O(250)$ Leute zu kochen, haben wir mit größtem Respekt angenommen und angesichts der handelsüblichen Küchenausstattung unserer Fachschaft mussten wir nicht nur einmal tief in die Trickkiste greifen. Das ein oder andere Mal wollten wir euch auch verhungern lassen \dots

Nichtsdestotrotz wurden leckere Rezepte vorgekocht, kalkuliert und optimiert, alles natürlich in vegan \underline{und} omniphor.  Auch eure zahlreichen Sonderwünsche werden wir versuchen, bestmöglich umzusetzen. Für besonders harte Anfragen können wir uns auf die kompetente Unterstützung von Profis verlassen. Ihr zuverlässlichen ZaPFika habt natürlich schon bei der Anmeldung alle nötige Informationen angegeben, aber für Spätentschlossene wird auch eine Liste der Inhaltsstoffe und Zutaten im Tagungsbüro ausliegen.

Wie alle Jahre wieder, wird es hauptsächlich das ewige Frühstück geben. Das Buffet ist im Goldenen Käfig aufgebaut und wird rund um die Uhr wieder aufgefüllt, damit auch der kleine  Hunger zwischendurch gestillt werden kann. Weil uns nur Brot mit Belag einfach nicht exzellent genug erscheint, haben wir euch einige besondere Mahlzeiten organisiert:
  \begin{itemize}
    \item \textbf{Freitagmittag}: Hier könnt ihr euch in unserer \textbf{Mensa} bedienen.
    \item \textbf{Donnerstagabend}: Vor der Chemie wird \textbf{gegrillt}. \\
      Natürlich vegan \underline{und} omniphor mit allem drum und dran.
  \end{itemize}
  Ihr seht: Für euer leibliches Wohl ist bestens gesorgt! \\

  Wenn euch doch mal unser Essen nicht exquisit genug ist, gibt es hier auf dem Campus auch die Möglichkeit
  anderweitig Essen zu gehen und es euch mal so richtig gut gehen lassen. Dafür können wir euch
  folgende Restaurants empfehlen:
  \begin{itemize}
  \item Café Botanik: Café, guter Imbiss, stud. Restaurant im hinteren Teil der Mensa
  \item Café Bellini: italienisches Restaurant (Im Neuenheimer Feld 371)
  %\item Bellini das Bistro: Bistro (Im Neuenheimer Feld 370)
  \item Konsumikon: Einkaufszeile mit Rewe, Aldi, Bäcker im Mathematikon (Berliner Str. 49)
  \item BräuStadl: bayrische Küche im Konsumikon (Teil des Mathematikons)
  \end{itemize}
