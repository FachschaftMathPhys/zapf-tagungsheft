% !TEX TS-program = pdflatex
% !TEX encoding = UTF-8 Unicode
% !TEX ROOT = main.tex

\section*{Exkursionen}

Diese finden am Freitag dem 1. Juni im Laufe des Vormittags statt.
Hier findet ihr Treffpunkt und Startzeit der Exkursionen, sowie kurze Beschreibungstexte.
Alle Zeiten sind so zu verstehen wie sie hier stehen, ohne a priori c.t. oder anderen akademischen Schabernack.

\subsection*{Wanderung auf dem Heiligenberg}
\textbf{Treffpunkt} Haupteingang Mathematikon\\
\textbf{Uhrzeit} 8:00\\
Auf dem Heiligenberg verläuft eines der beliebtesten Touristenziele Heidelbergs - der Philosophenweg. Neben der wundervollen Aussicht auf Heidelberg und das Neckartal, befinden sich auf dem Heiligenberg außerdem Überreste eines Klosters und die Thingstätte. Wer sich als Ausgleich zu den verschiedenen AKs für 4-5 Stunden an der frischen Luft aufhalten und bewegen möchte, der darf sich gern unserer Wanderung über den Philosophenweg Richtung Ziegelhausen, zum Gipfel des Heiligenbergs und via Alte Brücke zurück ins Neuenheimer Feld anschließen. Alle Teilnehmenden sollten nach eigenem Bedarf ein kleines Vesper und genug zu Trinken einpacken. An das Wetter angepasste Kleidung und festes Schuhwerk sind definitiv von Vorteil - wer aber 4h in Flip-Flops laufen kann, dem sei das gegönnt! 

\subsection*{Stadtführung}
\textbf{Treffpunkt} vor dem KIP-Foyer\\
\textbf{Uhrzeit} 9:15\\
Da wir der Meinung sind, dass Heidelberg eine ganz hübsche und ansehnliche Stadt ist, wollen wir sie euch natürlich auch zeigen. Zu diesem Zweck wollen wir eine kleine Stadtführung veranstalten und die wichtigsten Sehenswürdigkeiten abklappern. Wir starten im  Neuenheimer Feld und laufen von da aus am Neckar entlang über den Philosophenweg zur Altstadt. Dort besuchen wir zunächst das Schloss und besichtigen anschließend den Studentenkarzer. Zum Abschluss gibt es ein kurzes Ralleyspiel in der Altstadt für alle die Lust haben. Da einiges an Fußweg vor uns liegt, empfehlen wir passendes Schuhwerk. 

\subsection*{Körperwelten}
\textbf{Treffpunkt} Haupteingang Mathematikon\\
\textbf{Uhrzeit} 9:45\\
Seit September 2017 gibt es nun auch Körperwelten in Heidelberg. Hier werden beeindruckende Ganzkörperplastinate, genauso wie einzelne Organe des Körpers in etwa 200 echten menschlichen Präparaten gezeigt. Das Besondere an der Ausstellung in Heidelberg ist, dass sie das Thema \flqq Anatomie des Glücks\frqq in den Vordergrund stellt. Wegen der nicht unerheblichen Eintrittskosten werden \EUR{5} Eigenbeteiligung verlangt, der Rest wird subventioniert. 

\subsection*{Max-Planck-Institut für Kernphysik}
\textbf{Treffpunkt} Haupteingang Mathematikon\\
\textbf{Uhrzeit} 8:30\\
Das am Fuße Königsstuhl auf dem Boxberg gelegene MPIK betreibt physikalische Grundlagenforschung im Bereich Astroteilchenphysik und Quantendynamik. So finden sich hier z.B. Elektronenstrahl-Ionenfallen und ein ultrakalter Speicherring. Neben einem Einführungsvortrag wird es selbstverständlich auch eine Führung durch die Labore geben. Minderjährige Personen und Schwangere dürfen nicht an dieser Exkursion teilnehmen. 

\subsection*{Klosterhof Brauerei}
\textbf{Treffpunkt} Haupteingang Mathematikon\\
\textbf{Uhrzeit} 9:00\\
Was wäre eine ZaPF ohne Bier? Aus diesem Grund möchten wir einer regionalen Brauerei einen Besuch abstatten. 

\subsection*{European Molecular Biology Laboratory}
\textbf{Treffpunkt} Haupteingang Mathematikon \\
\textbf{Uhrzeit} 8:15\\
Das EMBL gehört zu den bekanntesten biologischen Forschungslaboren der Welt. Auf extraterritorialem Gebiet am Königstuhl liegend arbeiten hier auch Physikerinnen und Physiker, mit welchen wir bei dieser Exkursion ins Gespräch kommen werden. Neben einer Einführung ins EMBL und einer kurzen \flqq Lecture \frqq werden wir uns natürlich auch Labore anschauen.

\subsection*{SAP}
\textbf{Treffpunkt} Haupteingang Mathematikon\\
\textbf{Uhrzeit} 7:45\\
Die SAP SE hat ihren Sitz in Walldorf, südlich von Heidelberg. Dem Umsatz nach ist SAP der größte europäische Softwarehersteller sowie der weltweit viertgrößte. Tätigkeitsschwerpunkt ist die Entwicklung von Software zur Abwicklung sämtlicher Geschäftsprozesse eines Unternehmens wie Buchführung, Controlling, Vertrieb, Einkauf, Produktion, Lagerhaltung und Personalwesen. In der Rhein-Neckar Region ist SAP ein sehr beliebter Arbeitgeber für Akademiker*innen mit naturwissenschaftlicher Ausbildung. 

\subsection*{Uni-Archiv}
\textbf{Treffpunkt} Haupteingang Mathematikon\\
\textbf{Uhrzeit} 9:00\\
Wir werden gemeinsam durch die bunten Jahrhunderte springen und "spielerisch" die Geschichte der Universität entdecken: Disziplin, Autorität, Protest und das Leben zwischendurch. Mitzubringen sind Neugier und Lust an illustren vergangenen Zeiten. Der Geheimtipp unter den Exkursionen! 

\subsection*{Volume-Graphics}
\textbf{Treffpunkt} Haupteingang Mathematikon\\
\textbf{Uhrzeit} 9:20\\
Volume Graphics entwickelt führende Software zur Analyse und Visualisierung industrieller 3D-Computertomographiedaten. Weltweit nutzen mehr als 70\% der „Fortune Global 500“-Unternehmen in der Automobil- und Elektronikindustrie* sowie Unternehmen aus der Luft- und Raumfahrtindustrie Lösungen von Volume Graphics für Qualitätskontrolle, Messtechnik, Schadensanalyse und Produktentwicklung. Software wie die erweiterbare High-End-Lösung VGSTUDIO MAX helfen Unternehmen, möglichst alles über ihre Produkte herauszufinden – und das zerstörungsfrei. Die Basis dafür liefert die industrielle CT, denn ein CT-Scan durchleuchtet ein Bauteil komplett. Gegründet wurde das Unternehmen 1997 in Heidelberg als Spin-off der Ruprecht-Karls-Universität Heidelberg. Auch heute noch ist Heidelberg der Hauptsitz. Niederlassungen befinden sich in den USA, Japan, China und Singapur. Die Exkursion zum Hauptsitz von Volume Graphics verspricht Einblicke in die Arbeit des Unternehmens und der vielen hier tätigen Physiker*innen. Erfahre direkt von einem der Gründer, wie Volume Graphics vom universitären Start-up zu einem der führenden Anbieter industrieller CT-Software wurde. Lasse dir von anderen Physikern erklären, wie sie bei Volume Graphics ihrer Leidenschaft für 3D-Bildverarbeitung, physikalische Simulationen, Software-Architektur und vielem mehr nachgehen. Den Abschluss bildet ein Firmenrundgang, gefolgt von einem Imbiss und Zeit zum Austausch mit Mitarbeitern und anderen Teilnehmern. 

\subsection*{Friedrich-Ebert-Gedenkstätte}
\textbf{Treffpunkt} Haupteingang Mathematikon\\
\textbf{Uhrzeit} 9:30\\
Das Friedrich-Ebert-Haus erinnert an den 1871 dort geborenen ersten Reichspräsidenten der Weimarer Republik. Für Geschichtsinteressierte wird es eine Führung durch die Dauerausstellung "Vom Arbeiterführer zum Reichspräsidenten. Friedrich Ebert (1871-1925)" geben.

\subsection*{MPI für medizinische Forschung}
\textbf{Treffpunkt} Haupteingang Mathematikon\\
\textbf{Uhrzeit} TBA\\
Im Neuenheimer Feld gelegen forscht man hier interdisziplinär in den Bereichen Biomolekulare Mechanismen, Chemische Biologie, Zelluläre Biophysik und Optische Nanoskopie. 