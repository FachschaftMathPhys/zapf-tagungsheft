% !TEX TS-program = pdflatex
% !TEX encoding = UTF-8 Unicode
% !TEX ROOT = main.tex

\begin{lied*}{Reformgedicht}{Prof.\,Dr.\,Dieter Gromes\\zu seinem Abschied im Juli 2004}
Wenn die hohe Fakultät\\
Mittwoch nachmittags berät,\\
oder wenn's in Parlamenten\\
geht um Unis und Studenten,\\
wenn in wohlgeformten Sätzen\\
Journalisten Unsinn schwätzen,\\
scheint es immer sonnenklar,\\
nicht darf bleiben wie es war.\\
Alles ist doch antiquiert.\\
Alles wird jetzt reformiert.\\

Solch ein Tun, wer will's bestreiten\\
gab's auch schon in früh'ren Zeiten.\\
Junge Leute, zorngeladen,\\
gingen auf die Barrikaden,\\
und ertrotzten ungehalten\\
sich Reformen von den Alten.\\

Heut, man wird's kaum glauben wollen,\\
spielt man mit vertauschten Rollen\\
und erstaunt sieht man die Welt,\\
völlig auf den Kopf gestellt.\\

Junge halten artig still,\\
tun brav alles was man will.\\
Junge machen gern Physik,\\
und nicht Hochschulpolitik.\\
Junge möchten gern studier'n,\\
doch die Altern reformier'n.\\

Missionarische Gestalten\\
sind am Schalten und am Walten,\\
nie wird ihr Elan erkalten\\
nichts vermag sie auf zu halten,\\
alles weicht vor der geballten\\
Wucht der Revoluzzeralten.\\

Mag's die meisten auch verdrießen\\
keiner schafft's sich auszuschließen.\\
Morgens früh da steht man auf,\\
überdenkt den Tageslauf,\\
fragt, noch eh man sich rasiert:\\
"Was wird heut wohl reformiert?''\\

Und dann geht's gleich richtig los,\\
der Reformdruck riesengroß.\\
Fieberhaft sucht man nach Sachen,\\
die man anders könnte machen.\\
Alle sag'n wir sind so schlecht,\\
machen wir's den Leuten recht.\\
Gibt's noch was das funktioniert?\\
Das wird schleunigst reformiert.\\
Hat sich was schon lang bewährt?\\
Dann war's sicher ganz verkehrt.\\

Nur den Fortschritt nicht verpennen,\\
lasst uns alles umbenennen.\\
Chaos, Flops, und Supergaus,\\
keine Sau kennt sich mehr aus,\\
ständig wächst die Entropie,\\
die Reformer stört das nie.\\
Selbst wenn alles kollabiert\\
Es wird weiter reformiert.\\

Abends bist du ganz KO,\\
wirst erst langsam wieder froh,\\
denkst \glqq für heute ist's vollbracht\grqq ,\\
machst dich fertig für die Nacht,\\
und fällst müd wie eine Katze\\
schlaff auf die Reformmatratze.\\

Uns're Welt, sie wär verloren,\\
gäbs d'rin nichts als Professoren.\\
Denn man braucht doch auch Studenten,\\
und sei's nur für uns're Renten.\\
Doch die Frage ist im Kern:\\
\glqq Wieviel hätten wie denn gern?\grqq\\

Für die Fakultät, fürwahr,\\
war die Antwort immer klar.\\
\glqq Lasst uns auf der Lauer liegen,\\
möglichst viele woll'n wir kriegen.\grqq\\
Ganz genial war die Idee\\
der \glqq Physik macht Spass\grqq--CD.\\
\glqq Aufgepasst, das ist der Kick,\\
eilt herbei, studiert Physik!\grqq\\

Plötzlich, nein das ist nicht fair,\\
kommen die tatsächlich her.\\
Helle Panik, riesengroß\\
wie krieg'n wir die wieder los?\\

Macht euch darum keine Sorgen,\\
alles regelt sich schon morgen.\\
Nur schön weiter reformiert,\\
dann werd'n die schon dezimiert.\\
Und wenn das demnächst passiert,\\
ist schon heute garantiert,\\
dass man erst mal lamentiert,\\
und dann blitzschnell reagiert,\\
das Verfahren reformiert,\\
die Kriterien renormiert.\\

Ohne Zögern steigt man munter\\
dann vom hohen Ross herunter.\\
Und wenn alles das getan\\
geht es wieder flott voran,\\
selig lächelt der Dekan\\
alles läuft perfekt nach Plan,\\
und wir nehmen jeden an,\\
der bis fünfzig zählen kann.\\

Das Diplom, ganz ohne Frage\\
geht in seine letzten Tage.\\
War's auch weltweit angesehn,\\
so kann's doch nicht weiter gehn.\\
Denn wir woll'n jetzt alle Sachen\\
so wie in Kentucky machen.\\

Planungsgruppen weit und breit,\\
für die Forschung keine Zeit,\\
ganze Wissenschaftslegionen\\
sitzen in den Kommissionen.\\
Tausende von Arbeitsstunden\\
geh'n dahin in all den Runden.\\
Eine Riesenzahl von Weisen\\
plagt sich in den Arbeitskreisen,\\
und strickt, knirschend mit den Zähnen,\\
an den neuen Studienplänen.\\

Wird nach langem zähen Ringen\\
dann der große Wurf gelingen?\\
Oder droht nun ein Desaster\\
mit dem Bachelor und dem Master?\\

Sparen muss die Politik\\
sie versucht's mit manchem Trick.\\
Die Studenten, diese Plage,\\
sind doch eine Kostenfrage.\\
Deshalb wird es gern gesehn,\\
wenn sie schnellstens wieder gehn,\\
und uns die Studentenmassen\\
nach dem Bachelor gleich verlassen.\\

Auf ihre Leute, fix studiert,\\
dass mir keiner Zeit verliert.\\
Lehrstoff kräftig komprimiert,\\
kontrahiert und konzentriert,\\
reduziert und minimiert,\\
auch wenn man nicht viel kapiert.\\

Internationalisiert,\\
keiner fragt was man verliert.\\
Alles dann perfekt fixiert,\\
und gleich wieder variiert,\\
nochmal Stoff eliminiert,\\
nochmal anders arrangiert,\\
noch ein bißchen umfrisiert,\\
noch 'ne Änd'rung produziert,\\

Dann das Ganze permutiert,\\
und gleich noch mal reformiert,\\
ja nicht zwischendurch pausiert,\\
nicht nach links und rechts gestiert\\
schnell noch was dahingeschmiert\\
schwupps den Bachelor abkassiert.\\

Der Erfolg ist grandios,\\
ihr seid früher arbeitslos.\\

Niemand weiss wie's weiter geht\\
und auch ich bin kein Prophet.\\
Doch ich wag' schon mal ganz lose\\
jetzt die folgende Prognose:\\

Wenn in Deutschland und der Welt\\
niemand viel von all dem hält,\\
wenn die Chefs in den Betrieben\\
diesen Abschluss gar nicht lieben,\\
wenn im fernen Amiland\\
nichts von dem wird anerkannt,\\
wird man deshalb nicht gleich bang\\
und man zögert auch nicht lang,\\
schafft etwas von hohem Rang\\
einen \glqq Aufbaustudiengang\grqq .\\
Sucht dann noch in diesem Rahmen\\
einen eindrucksvollen Namen.\\
Und ein toller wär gewiss:\\
\glqq Aufbaustudium Dipl. Phys.\grqq.\\

Ganz speziell die jungen Leute\\
haben's gar nicht einfach heute.\\
Wildes Forschen ist tabu,\\
alles regelt die EU.\\

Willst du an der Uni bleiben\\
musst du manchen Antrag schreiben.\\
Antwort ist auf hundert Fragen\\
in die Kästchen einzutragen.\\
Haarklein will man schon erfahren\\
dein Ergebnis in drei Jahren,\\
wie dereinst, in welcher Eichung,\\
ausseh'n soll die neue Gleichung.\\
Und noch, bitteschön, in Strenge,\\
die exakte Lösungsmenge.\\

Newton und Madame Curie\\
konnten so was leider nie.\\
Heisenberg und auch Niels Bohr\\
stünden da ganz dumm davor.\\
In der ganzen Wissenschaft\\
hat das niemand noch geschafft.\\
Denn wenn man's schon vorher weiss\\
ist's bedeutungslose Schmarrn.\\

Mach aus all dem dir nichts draus\\
füll den Schwachsinn trotzdem aus.\\
Macht auch alles keinen Sinn\\
schreib halt einfach was dahin.\\
Lass den Leuten ihren Spaß\\
und erfinde irgendwas.\\

Doch pass auf, das Formular\\
bringt beträchtliche Gefahr,\\
und du bist bereits verdammt\\
falls das Ding aus Brüssel stammt.\\
Denn hast du in Zeile sieben\\
aus Versehen dich verschrieben,\\
brauchst du bei den andern Zeilen\\
gar nicht länger zu verweilen.\\
Uns're Eurobürokraten\\
strafen derlei Missetaten.\\
Du hast alles schon vermasselt\\
und bist hochkant durchgerasselt.\\

Pisa und auch Toll Collect\\
haben die Nation erschreckt.\\
Hinten woll'n wir doch nicht sein,\\
doch wie kommt man vorne rein?\\

Laut erschallt des Kanzlers Wort\\
man vernimmt's an jedem Ort.\\
Um ganz vorne mit zu bieten\\
da bedarf es der Eliten.\\

Alle finden das ganz fein,\\
alle woll'n Eliten sein.\\
Alle Unis weit und breit\\
stehn begeistert gleich bereit.\\

Überall in deutschen Gauen\\
kann man nur Eliten schauen.\\
Von den Alpen bis zum Meere\\
blickst du auf Elitenheere.\\
Tausend Meilen magst du gehn\\
du wirst nur Eliten sehn.\\
Denn sogar die letzte Niete\\
zählt sich fortan zur Elite.\\

Für archaische Strukturen\\
laufen ab die alten Uhren.\\
Planlos wursteln geht nicht mehr,\\
ein Strukturplan der muss her.\\
Mit viel Mühe, Zeit und Kraft\\
hat man's schließlich auch geschafft.\\
Als vollendet war der Frust,\\
wirft man stolz sich in die Brust.\\
Alle war'n des Lobes voll,\\
fanden den Strukturplan toll.\\
Denn mit solch perfektem Plan,\\
bricht man für die Zukunft Bahn.\\

Noch die Druckerschwärze frisch,\\
ist's schon wieder weg vom Tisch.\\
\glqq Nur Physik\grqq, so schallt's im Chor,\\
\glqq lockt doch keinen Hund hervor.\grqq\\
Mal ganz schnell zusammensetzen\\
und die ganze Welt vernetzen,\\
die Strukturen reformieren,\\
und was Neues etablieren,\\
dass ihr jetzt auch alles wisst,\\
was das Feld der Zukunft ist:\\

Anglo-Medien-Nano-Statik\\
Kosmo-Bio-Informatik\\
Gen-Astrologie-Dynamik\\
Quanten-Umwelt-Chip-Keramik\\
Quintessenz-Magnet-Neuronen\\
Medizin-Quasar-Ionen\\
Galaxie-Synapsen-Kerne\\
Q-bit-Chromosomen-Sterne,\\
und noch mancherlei dergleichen\\
wird uns bald zum Ruhm gereichen.\\

Hoffnung scheint nicht mehr zu sehen\\
der Reformwut zu entgehen.\\
Denn schon in dem großen Buche\\
wird berichtet von dem Fluche.\\
Dort im Neuen Testament\\
steht der Satz den jeder kennt.\\
Nichts wird fürderhin bestehn,\\
alles wird zugrunde gehn,\\
und, wie die Propheten schreiben,\\
kein Stein auf dem anderen bleiben.\\

Ein Stein aber aus den Mauern\\
wird die Zeiten überdauern.\\
Albert Einstein hat gelehrt,\\
wir seh'n alles ganz verkehrt.\\
Leute seid nicht so naiv,\\
alles ist doch relativ.\\
Ob bescheuert oder gut,\\
alter oder neuer Hut,\\
ob verkrustet, ob extrem\\
hängt nur am Bezugssystem.\\

Seht, das Weltall expandiert,\\
während ihr hier reformiert.\\
Wie spricht Heraklit der Weise?\\
\glqq Alles dreht sich doch im Kreise,\\
und man kommt nach Tag und Jahr\\
dort an wo man schon mal war.\grqq\\

Alle freun sich dann enorm\\
über die Reformreform,\\
klatschen fröhlich in die Hände,\\
jetzt ist das Gedicht zu Ende.\\
\end{lied*}

